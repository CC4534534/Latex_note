we wish to find an orthogonal matrix A that minimizes the loss function
L(A)=\frac{1}{2} \sum_{i=1}^{n}a_{i} \left |\hat{W_{i} } -A\hat{V_{i}}     \right | ^2
where a_{i} are a set of non-negative weights such that sum_{i=1}^{n}a_{i=1}, hat{V_{i}} are nonparallel reference vectors, and hat{W_{i} } are the corresponding observation vectors.
The gain function is define by 
g(A)=1-L(A)=\sum_{i=1}^{n}a_{i}  \hat{W_{i}^{T}}A \hat{V_i}  
The loss function L(A) is at its minimum when the gain function g(A) is at its maximum. The gain function can be reformulated as:
g(A)=\sum_{i=1}^{n}a_{i}tr(  \hat{W_{i}^{T}}A \hat{V_i}  )=tr(AB^T)
where tr is the trace of a matrix, and B is the attitude profile matrix:
B=\sum_{i=1}^{n}a_{i}\hat{W_i}\hat{V^{T}_{i}}
The quaternion \bar{q}  representing a rotation is defined by Shuster as:
\bar{q} = \begin{bmatrix}Q
\\q
\end{bmatrix}=\begin{bmatrix}\hat{X}sin(\frac{\theta }{2} ) 
\\cos(\frac{\theta }{2} ) 
\end{bmatrix}
where hat{X} is the axis of rotation, and \theta is the angle of rotation about hat{X}
Because the quaternion works as a versor, it must satisfy:
\bar{q}^T\bar{q}=\left | Q \right | ^2+q^2=1
The attitude matrix A is related to the quaternion by :
\mathbf{A}(\overline{\mathbf{q}})=\left(q^{2}-\mathbf{Q} \cdot \mathbf{Q}\right) \mathbf{I}+2 \mathbf{Q} \mathbf{Q}^{T}+2 q\lfloor\mathbf{Q}\rfloor_{\times}
where I is the identity matrix, and \lfloor\mathbf{Q}\rfloor_{\times} is the antisymmetric matrix of Q, a.k.a. the skew-symmetric matrix:
\lfloor\mathbf{Q}\rfloor_{\times}=\begin{bmatrix}
0  & Q_{3} & -Q_{2}\\
-Q_{3}  & 0 & Q_{1}\\
Q_{2} & -Q_{1} &0
\end{bmatrix}
Now the gain function can be rewritten again, but in terms of quaternions:
g(\bar{q})=(q^2-Q \cdot Q)trB^{T}+2tr(QQ^TB^T)+2qtr(\lfloor\mathbf{Q}\rfloor_{\times}B^T)
A further simplification gives:
g(\bar{q})=\bar{q}^TK\bar{q}
where the 4*4 matrix K is given by:
K=\begin{bmatrix}S-\sigma I&Z
\\
Z^T  &\sigma
\end{bmatrix}
using the helper values:
\sigma = trB
S = B+B^T
Z =\sum_{i=1}^{n}a_{i}\hat{W_i}\times \hat{V_{i}}
A new gain function g'(\bar{q})with Lagrange multipliers is defined:
g'(\bar{q})=\bar{q}^TK\bar{q}-\lambda\bar{q}^T\bar{q}
It is verified that K\bar{q}=\lambda \bar{q}. Thus, g(\bar{q}) will be maximized if \bar{q}_opt is chosen to be the eigenvalue \lambda:
\lambda = \sigma + Z \codt Y
where Y is the Gibbs vector, a.k.a the Rodrigues vector, defined as:
Y = \frac{Q}{q}=\hat{X}tan\frac{\theta}{2}
rewriting the quaternion as:
\bar{q} = \frac{1}{\sqrt{1+|Y|^2} }=\begin{bmatrix}Y
\\1
\end{bmatrix}
Y and \bar{q} are representations of the optimal attitude solution when \lambda is equal to \lambda_max , leading to an equation for the eigenvalues:
\lambda = \sigma + Z^{T} \frac{1}{(\lambda+\sigma)I-S}Z
which is equivalent to the characteristic equation of the eigenvalues of K
With the aid of Cayley-Hamilton theorem we can get rid of the Gibbs vector to find a more convenient expression of the characteristic equation:
\lambda^{4}-(a+b)\lambda^{2}-c\lambda+(ab+c\sigma-d)=0
where:
a=\sigma^{2}-\kappa
b = \sigma^{2}+Z^{T}Z
c = \bigtriangleup +Z^{T}SZ
d = Z^{T}S^{2}Z
\sigma = \frac{1}{2}trS
\kappa = tr(adj(S))
\bigtriangleup = det(S)
To find \lambda we implement the Newton-Raphson method using the sum of the weights a_{i} (in the beginning is constrained to be equal to 1) as a starting value.
\lambda_{t+1}\gets \lambda_{t}-\frac{f(\lambda)}{f'(\lambda)}=\lambda_{t}-\frac{\lambda^{4}-(a+b)\lambda^{2}-c\lambda+(ab+c\sigma-d)}{4\lambda^{3}-2(a+b)\lambda-c}
For sensor accuracies better than 1 arc-min(1 degree) the accuracy of a 64-bit word is exhausted after only one iteration.
Finally, the optimal quaternion describing the attitude is found as:
\bar{q}_{opt}=\frac{1}{\sqrt{\gamma^{2}+|X|^{2}}}\begin{bmatrix}X
\\ \gamma
\end{bmatrix}
with:
X = (\alpha I+\beta S+S^{2})Z
\gamma = (\lambda+\sigma)\alpha-\bigtriangleup
\alpha = \lambda^{2}-\sigma^{2}+\kappa
\beta = \lambda - \sigma
This solution can still lead to an indeterminant result if both \gamma and X vanish simultaneously. \gamma vanishes if and only if the angle of rotation is equal to \pi, even if 
X does not vanish along.


---------------------------------------------------------------------------------------------------------
In 1965 Grace Wahba came up with a simple, yet very intuitive, way to describe the problem of finding a rotation between two coordinate systems.

Given a set of N vector measurements u in the body coordinate system, an optimal attitude matrix A would minimize the loss function:
L(A)=\frac{1}{2} \sum_{i=1}^{N}w_{i} \left |u_{i} -A\hat{v_{i}}     \right | ^2
where u_{i} is the i-th vector measurement in the body frame, v_{i} is the i-th vector in the reference frame, and w_{i} are a set of N nonnegative weights for each observation. This famous formulation is known as Wahba’s problem.
A first elegant solution was proposed by [Dav68] that solves this in terms of quaternions, yielding a unique optimal solution. The corresponding objective function is defined as:
g(\mathbf{A})=1-L(\mathbf{A})=\sum_{i=1}^{N} w_{i} \mathbf{U}^{T} \mathbf{A} \mathbf{V}
The objective function is at maximum when the loss function L(A) is at minimum. The goal is, then, to find the optimal attitude matrix A, which maximizes g(A). We first notice that:
g(\mathbf{A})=\sum_{i=1}^{N} w_{i} \operatorname{tr}\left(\mathbf{U}_{i}^{T} \mathbf{A} \mathbf{V}_{i}\right)=\operatorname{tr}\left(\mathbf{A B}^{T}\right)
where \operatorname{tr} denotes the trace of a matrix, and B is the attitude profile matrix:
\mathbf{B}=\sum_{i=1}^{N} w_{i} \mathbf{U V}
Now, we must parametrize the attitude matrix in terms of a quaternion q [Ler78] :
\mathbf{A}(\mathbf{q})=\left(q_{w}^{2}-\mathbf{q}_{v} \cdot \mathbf{q}_{v}\right) \mathbf{I}_{3}+2 \mathbf{q}_{v} \mathbf{q}_{v}^{T}-2 q_{w}\lfloor\mathbf{q}\rfloor_{\times}
where I_{3} is a 3x3 identity matrix, and the expression \lfloor\mathbf{q}\rfloor_{\times} is the skew-symmetric matrix of a vector q.
The objective function, in terms of quaternion, becomes:
g(\mathbf{q})=\left(q_{w}^{2}-\mathbf{q}_{v} \cdot \mathbf{q}_{v}\right) \operatorname{tr} \mathbf{B}^{T}+2 \operatorname{tr}\left(\mathbf{q}_{v} \mathbf{q}_{v}^{T} \mathbf{B}^{T}\right)+2 q_{w} \operatorname{tr}\left(\lfloor\mathbf{q}\rfloor_{\times} \mathbf{B}^{T}\right)
A simpler expression, using helper quantities, can be a bilinear relationship of the form:
g(\mathbf{q})=\mathbf{q}^{T} \mathbf{K q}
where the 4x4 matrix K is built with:
\mathbf{K}=\left[\begin{array}{cc}
    \sigma & \mathbf{z}^{T} \\
    \mathbf{z} & \mathbf{S}-\sigma \mathbf{I}_{3}
    \end{array}\right] \\
    using the intermediate values:
    \sigma=\operatorname{tr} \mathbf{B} \\
    \mathbf{S}=\mathbf{B}+\mathbf{B}^{T} \\
    \mathbf{z}=\left[\begin{array}{l}
    B_{23}-B_{32} \\
    B_{31}-B_{13} \\
    B_{12}-B_{21}
    \end{array}\right] \\
The optimal quaternion \hat{q}, which parametrizes the optimal attitude matrix, is an eigenvector of K. With the help of Lagrange multipliers, g(q) is maximized if the eigenvector corresponding to the largest eigenvalue \lambda is chosen.


